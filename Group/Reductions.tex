\documentclass[]{article}
\usepackage[left=0.5in,right=0.5in,top=0.5in,bottom=0.8in]{geometry}
\usepackage{commath,amsmath,amssymb,amsfonts}
\usepackage{algorithmic}
\usepackage{graphicx}
\graphicspath{{./images/}}
\usepackage{textcomp}
\usepackage{xcolor}
\usepackage[backend=biber, sorting=none]{biblatex}
\usepackage[hidelinks]{hyperref}

%uncomment below line to use reference file
%\addbibresource{references.bib}


\begin{document}

\section*{Reductions}
The first reduction described in (P1) is a reduction from \textsc{NASH} to the \textsc{END OF THE LINE} problem given in (P1) in order to prove \textsc{NASH} is in \textbf{PPAD}. This reduction uses Brouwer's fixed point theorem (CITE), which put simply means that a mapping in the Euclidean space of a compact and convex subset back onto itself must have some fixed point that does not change in the mapping $F$. This gives rise to the problem \textsc{BROUWER}, this is simply when given $F$ and subset find a fixed point. Two inconveniences are discussed in (P1) which are how to specify $F$ and how to deal with fixed points that are not rational numbers. These are solved by first restricting the subset to the unit cube and by assuming that $F$ is given by some tractable algorithm which can map from the unit cube to the correct value. To deal with irrational fixed points $F$ is assumed to obey a simple Lipschitz condition:
\begin{equation}
	\text{for all}x_1,x_2\in[0,1]^m:d(F(x_1),F(x_2))\leq K\cdot d(x_1,x_2)
	\label{eq:1}
\end{equation}
this simply means that the distance between result for 2 points is bounded by $K$ the Lipschitz constant of $F$ and distance between the 2 points. This is done so the domain can be discretized. As with \textsc{NASH} this means approximate fixed points bounded by some constant $\epsilon$ can be found. A reduction exists from \textsc{BROUWER} to \textsc{END OF THE LINE}(P1). This is done by dividing the unit square into a mesh of small triangles (triangulation). The vertices of the triangles are coloured based on the angle $F$ maps it in from the horizontal. This 3-colouring satisfies a property that red cannot appear on the lower side of the square, blue cannot appear on the left side of the square, and the other 2 sides cannot have yellow. Sperner's Lemma (CITE) states that any colouring obeying this property must have a triangle with all 3 colours. By \autoref{eq:1} the vertices of such a triangle are fixed points. By having all triangles with a red and yellow vertex form a directed graph in which an edge exists from a triangle to another if they share an edge from red to yellow clockwise in the first triangle. Such a graph consists only of paths and cycles (P1). An assumption can be made the left of the square only has one edge between yellow and red. This edge is part of a triangle that may be 3-coloured or if not must form a path that ends at a 3-coloured triangle meaning at least 1 3-coloured triangle must exist. This kind of graph is an instance of \textsc{END OF THE LINE}(P1). \textsc{NASH} can be trivially reduced to \textsc{BOUWER} as a function can be defined maps from the set of strategies to itself, based on payoff improvement. It is clear that a fixed point under such a function must correspond to a Nash equilibrium as there is no way to improve from it. Therefore this function and set of strategies form an instance of \textsc{BOUWER} consequently \textsc{NASH} is reducible to \textsc{END OF THE LINE}.\par
To prove \textsc{NASH} is \textbf{PPAD}-complete (P1) presents a set of reductions from an instance of \textsc{END OF THE LINE} can be reduced to \textsc{NASH}. First \textsc{END OF THE LINE} is reduced to \textsc{BROUWER} (P1), the unit cube is again used and it is subdivided into a mesh of smaller cubes (cubelets)  each with a grid point in its centre. These grid points can then be assigned 1 of 4 colours $[0-4]$ based on the vector $F(x)-x$. The vectors determining what colour is chosen are set so $F(x)-x$ is only near $0$ if $x$ is near all 4 colours. To convert an instance $G$ of \textsc{END OF THE LINE} to such a cube each vertex of the \textsc{END OF THE LINE} graph is placed either in bottom or top left edges of the cube. This gives rise to a special property that if an edge exists in $G$ between 2 vertices then a sequence of grid points will be coloured 1,2, or 3. It is possible to create an arrangement of colours such that all 4 are only adjacent grid points (corresponding to an approximate fixed point) in an area of the cube that corresponds to a solution to \textsc{END OF THE LINE} and is thus \textbf{PPAD}-complete. Next by reducing \textsc{BROUWER} to \textsc{NASH} it can be shown \textsc{NASH} is also \textbf{PPAD}-complete. The previous reduction is such that all $F$'s can be computed using circuits consisting of addition, multiplication, and comparison operators forming a dataflow graph (P1). Each node of this graph can be converted into its own game which can then be composed with the other nodes games forming a larger game representing the full circuit result. Each of the mini games has 4 players $w$, $x$, $y$, $z$. Players can take either a "go" or "stop" action such that $z$'s action represents the result of performing the node operator on the probabilities of $x$ and $y$'s strategies, $w$ acts as a mediating node between $x$, $y$, and $z$ (P1). Payoffs are then defined for the players such that, for multiplication, at any Nash equilibrium the probability $z$ picks "go" is the product of the $x$ and $y$ probabilities (P1), payoffs can similarly be chosen for addition and multiplication by a constant. To represent a $F$ using this method, the location of a point $x$ in the unit cube can be represented by 3 players each choosing "go" with probability equal to the corresponding coordinate in the cube. Then by chaining together enough of the mini games described above $F$ can be calculated as the 3 output players whose "go" probabilities are the coordinates of $F(x)$. It is possible to assign suitable payoffs such that any Nash equilibrium results in the input players and output players having the same "go" probabilities which corresponds to a fixed point. In order to avoid contradicting Nash's theorem $F$ must be calculated as the average of a grid around the point in the cube due to \textbf{the brittle comparator problem}(P1) giving an approximate fixed point. The game is by construction a graphical game (P1) and can be simulated as a three-player normal form game by having the 3 players represent $m$ nodes each that never play against each other in the dataflow graph or share a node that they both play against. Each of the players nodes don't compete with each other so there are no conflicts of interest, this means any Nash equilibrium in this 3 player game corresponds to one in the original game. A final necessary adjustment is to have the players compete in a rock-paper-scissors game to ensure the mixed-strategy profiles are balanced. The resulting 3 player game is efficient to compute and completes the reductions from \textsc{BROUWER} to \textsc{NASH} so \textsc{NASH} is \textbf{PPAD}-complete.
%uncomment below line to print bilbliography
%\printbibliography
\end{document}
